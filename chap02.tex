\chapter{Implementation}

As we repeatedly mentioned in the first chapter, the design we chose should allow multiple implementations for each module. The implementations provided in this thesis are designed for use-cases ranging from beginner user that requires only core functionality to advanced users who can utilize full module distribution. The resulting implementation therefore offers a monolithic application as well as separate modules.
\par
In this chapter we will discuss the implementation details of every module. At the beginning we will explain our programming language choices. We will add a short description for every library that we used and a reason why we preferred it to others.

%%-----------------------------------------------------------------------------------------
%% SECTION
%%-----------------------------------------------------------------------------------------
\section{Programming Language}

First decision we had to take was to select a suitable programming language. We had to follow the main guidelines that we have set earlier. That is, we had to find a way to support multiple platforms, while in the same time we needed flexible enough language to handle our modular design and fast enough to help us with scalability.
\par
Even though our modules are separable and therefore each of them could be written in a different language, we wanted to reduce the amount of used languages to minimum, ideally just one or two for the sake of simplicity. Furthermore, we focused on languages that would allow us to write one implementation for all platforms with the minimum of tweaks. Last but not least, we wanted a language that we had at least some experience with.
\par
We have narrowed the choice to three possible main programming languages for our implementation: C++, C\# and Java. All of them are major languages, have cross-platform capabilities and are well-established with enough libraries for all our features. We excluded popular languages Python and JavaScript due to their nature as scripting languages and therefore not providing enough performance for a bigger scale project.

%%-----------------------------------------------------------------------------------------
\subsection{GUI struggles}

The weak spot of all three languages is graphical user interface~(GUI) necessary for our Manager module. We would like the module to be modern and visually attractive and common GUI frameworks are usually either platform-specific, difficult to work with or have too many limitations. The exception might be
Qt~\citep{qt}, which is a very capable GUI toolkit written in C/C++, plus there is a Java wrapper for this library available.
\par
In the past few years, web applications have become very popular. Increasing performance of personal computers and development in web technologies as well as rise of single-page application frameworks allowed the creation of applications running almost entirely in a web browser. Nowadays, many people use only web browser on their computer where they have access to all sorts of web applications for any purpose. This has also changed their perception of how user interface should look like. Following this trend there are some desktop applications written entirely using web technologies, enclosed in a web browser wrapper.
\par
We decided to adopt above-mentioned approach for our Manager module. There is an open-source project written in C++ called Chromium Embedded Framework~(CEF). CEF focuses on facilitating embedded browser use cases in third-party applications~\citep{cef}. It means that we can create GUI using HTML5 and JavaScript relying on flexibility, cross-platform support and relative simplicity that they provide compared to other options. CEF offers a way to call C++ code from JavaScript so we can further customize its behavior as a proper desktop application.
\par
It is important to note that this decision was influenced by the desire to provide both modular as well as monolithic-looking application. Web GUI can be written once and then be utilized as a stand-alone module running on any web browser or wrapped as desktop application using CEF. Moreover, we can extend it using C++ to bundle it up with the remaining modules and create all-in-one application. Finally, it integrates well with our Database module as Firebase provides its SDK in JavaScript.

%%-----------------------------------------------------------------------------------------
\subsection{Final decision}

Having chosen a programming language and additionally a framework for Manager module, it remains to choose it for Player and Fileserver modules. Out of the three preselected languages we made C++ our final decision.
Both other languages would be equally capable for the task, but C++ provides more performance and more control of memory management. Furthermore, it does not rely on a virtual machine and is already utilized within our Manager module. Lastly, these two modules contain a lot of low-level programming, what makes C++ a preferable candidate and a lot of related libraries are written directly in C/C++.

%%-----------------------------------------------------------------------------------------
%% SECTION
%%-----------------------------------------------------------------------------------------
\section{Fileserver Module}
\todo{add short intro about how we designed it}

\subsection{Libraries used}
\todo{write about cpprestsdk and sqlite}

\subsection{Data Storage}
\todo{say how we store data and explain sql structure}

\subsection{Rest API}
\todo{say how we implemented the api, used HTTP methods, asynchronous tasks}

\subsection{Security}
\todo{discuss a bit about security and what were the options and what did we choose to follow}

%%-----------------------------------------------------------------------------------------
%% SECTION
%%-----------------------------------------------------------------------------------------
\section{Player Module}
\todo{add short intro about how we designed it}

\subsection{Libraries used}
\todo{write about boost, ffmpeg and portaudio}

\subsection{File Caching}
\todo{say how we store and cache data}

\subsection{Communication Protocol}
\todo{say how to communicate with player}

\subsection{Audio Decoding and Playback}
\todo{explain how ffmpeg communicates with portaudio and all other things}

\subsection{Security}
\todo{discuss a bit about security and what were the options and what did we choose to follow}

%%-----------------------------------------------------------------------------------------
%% SECTION
%%-----------------------------------------------------------------------------------------
\section{Manager Module}
\todo{add short intro about how we designed it}

\subsection{Libraries used}
\todo{write about React, CEF, MaterialUI, intl, and maybe axios}

\subsection{Design Principles}
\todo{explain what principles we held during implementation, dumb and smart components, global state, action creators, reducers, routes}

%%-----------------------------------------------------------------------------------------
%% SECTION
%%-----------------------------------------------------------------------------------------
\section{Realtime Database Module}
\todo{add short intro about how we designed it}

\subsection{Structure}
\todo{describe firebase structure}