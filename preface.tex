\chapter*{Introduction}
\addcontentsline{toc}{chapter}{Introduction}

Listening to music is one of the most popular ways to relax for many people and music playing in the background is one of the most common things that makes people feel better and more relaxed at any place, whether it is in a shopping centre, in a restaurant, in a waiting room at hospital or in your car.
\par
While you can choose what you want to listen to in your car, it is impossible in most of the other places. The only option for people to choose what they want to listen to in a public place are jukeboxes in bars. The technology has moved in a fast pace since they were introduced, but jukeboxes remain big boxes taking up space, eating coins when most of the people pay by card or by smart phone. Many places and venues therefore offer no choice of what's being played.
\par
The goal of this thesis is to design a music player where playback can be remotely controlled by multiple users. Application operator may create a music library from songs stored in local files (such as MP3) and organize these songs in playlists. Regular users may explore playlists and enqueue songs into the playback queue. The application is ready for integration with mobile technologies, so regular users may control the playback via a mobile application frontend.

%%-----------------------------------------------------------------------------------------
%% SECTION
%%-----------------------------------------------------------------------------------------
\section*{Target Group}
\addcontentsline{toc}{section}{Target Group}

As with every piece of commercial software, we would like to make our target group the widest possible to attract a lot of potential users. This is reflected in some of the decisions we had to make during design and implementation steps.
\par
There are many places that would benefit from features that this application offers. From small private home parties where anyone can become a DJ, through waiting rooms in hospitals or hairdressers where people often spend a lot of time waiting without any other entertainment besides music, to restaurants and bars where music is an essential part of a good atmosphere and completes the whole feel of the place.
\par
We would not like to disallow anyone from this range of potential customers to use this application. The final product should offer platform independence, so everyone can choose what suits them best, modularity, so everyone can customize it according to their needs, and scalability, so no one will be held back once they need to grow.

%%-----------------------------------------------------------------------------------------
%% SECTION
%%-----------------------------------------------------------------------------------------
\section*{Legal Perspective}
\addcontentsline{toc}{section}{Legal Perspective}

The complexity of laws in Czech Republic might have slowed down the creation of similar applications and certainly influences what services are offered by commercially available applications.
\par
This thesis focuses on the process of designing and implementing such application and disregards any past, present or future laws and legal directives that apply on this matter as it goes beyond it's topic.

%%-----------------------------------------------------------------------------------------
%% SECTION
%%-----------------------------------------------------------------------------------------
\section*{Current Situation}
\addcontentsline{toc}{section}{Current Situation}

There are currently multiple companies offering music solutions to public places. Most of them focus on providing music content and preparing tailor-made playlists created by professionals in return for a payment. They offer no interaction to anyone else than the administrator of the application. More details can be found in chapter \todo{Add chapter link}

\todo{Add a section about the structure of the thesis}

\section*{Previous Work}
\addcontentsline{toc}{section}{Previous Work}

This thesis is a follow-up on my semestral work. It was a proof of concept that a music player remotely controlled by multiple users might be useful and enabled me to test important technologies that gave basis to the solution described in this thesis. It was a music application capable of remote control written in JavaScript and HTML5 that was running in Electron. Electron is a framework for creating native applications with web technologies like JavaScript, HTML, and CSS~\citep{electron}. It allowed me to verify the technology without having to implement too many details myself.




